%-------------------------
% Resume in Latex
% Based on: Jake Gutierrez's template (https://github.com/jakegut/resume)
% License : MIT
%------------------------

\documentclass[letterpaper,11pt]{article}

\usepackage{latexsym}
\usepackage[empty]{fullpage}
\usepackage{titlesec}
\usepackage{marvosym}
\usepackage[usenames,dvipsnames]{color}
\usepackage{verbatim}
\usepackage{enumitem}
\usepackage[hidelinks]{hyperref}
\usepackage{fancyhdr}
\usepackage[english]{babel}
\usepackage{tabularx}
\input{glyphtounicode}

\pagestyle{fancy}
\fancyhf{}
\fancyfoot{}
\renewcommand{\headrulewidth}{0pt}
\renewcommand{\footrulewidth}{0pt}

% Adjust margins
\addtolength{\oddsidemargin}{-0.5in}
\addtolength{\evensidemargin}{-0.5in}
\addtolength{\textwidth}{1in}
\addtolength{\topmargin}{-.5in}
\addtolength{\textheight}{1.0in}

\urlstyle{same}

\raggedbottom
\raggedright
\setlength{\tabcolsep}{0in}

% Sections formatting
\titleformat{\section}{
  \vspace{-4pt}\scshape\raggedright\large
}{}{0em}{}[\color{black}\titlerule \vspace{-5pt}]

% Ensure that generated pdf is machine readable/ATS parsable
\pdfgentounicode=1

%-------------------------
% Custom commands
\newcommand{\resumeItem}[1]{
  \item\small{
    {#1 \vspace{-2pt}}
  }
}

\newcommand{\resumeSubheading}[4]{
  \vspace{-2pt}\item
    \begin{tabular*}{0.97\textwidth}[t]{l@{\extracolsep{\fill}}r}
      \textbf{#1} & #2 \\
      \textit{\small#3} & \textit{\small #4} \\
    \end{tabular*}\vspace{-7pt}
}

\newcommand{\resumeSubSubheading}[2]{
    \item
    \begin{tabular*}{0.97\textwidth}{l@{\extracolsep{\fill}}r}
      \textit{\small#1} & \textit{\small #2} \\
    \end{tabular*}\vspace{-7pt}
}

\newcommand{\resumeProjectHeading}[2]{
    \item
    \begin{tabular*}{0.97\textwidth}{l@{\extracolsep{\fill}}r}
      \small#1 & #2 \\
    \end{tabular*}\vspace{-7pt}
}

\newcommand{\resumeSubItem}[1]{\resumeItem{#1}\vspace{-4pt}}

\renewcommand\labelitemii{$\vcenter{\hbox{\tiny$\bullet$}}$}

\newcommand{\resumeSubHeadingListStart}{\begin{itemize}[leftmargin=0.15in, label={}]}
\newcommand{\resumeSubHeadingListEnd}{\end{itemize}}
\newcommand{\resumeItemListStart}{\begin{itemize}}
\newcommand{\resumeItemListEnd}{\end{itemize}\vspace{-5pt}}

%-------------------------------------------
%%%%%%  RESUME STARTS HERE  %%%%%%%%%%%%%%%%%%%%%%%%%%%%


\begin{document}

%----------HEADING----------
\begin{center}
    \textbf{\Huge \scshape Aslan Abdinabiev} \\ \vspace{1pt}
    \small (010) 9506-9707 $|$ \href{mailto:aslan@uos.ac.kr}{\underline{aslan@uos.ac.kr}} $|$
    Seoul, Korea $|$
    \href{https://github.com/Aslan7197}{\underline{github.com/Aslan7197}} $|$
    \href{https://scholar.google.com/citations?user=W-YakaQAAAAJ&hl=en}{\underline{Google Scholar}} $|$
    \href{https://Aslan7197.github.io/resume}{\underline{Portfolio}}
\end{center}


%-----------SUMMARY-----------
\section{Summary}
 \begin{itemize}[leftmargin=0.15in, label={}]
    \small{\item{
    PhD candidate building tools that automatically find and fix bugs in code using large language models. My work has improved automated repair success rates by 37\% over existing methods. Currently in the thesis stage of my PhD (expected 2027). Open to part-time research, engineering, or development positions, with interest in transitioning to full-time after graduation.
    }}
 \end{itemize}


%-----------EDUCATION-----------
\section{Education}
  \resumeSubHeadingListStart
    \resumeSubheading
      {University of Seoul}{Seoul, Korea}
      {PhD in Software Engineering}{2024 -- Present}
      \resumeItemListStart
        \resumeItem{Research focus: Automated Program Repair using Large Language Models}
      \resumeItemListEnd
    \resumeSubheading
      {University of Seoul}{Seoul, Korea}
      {M.Sc. in Software Engineering}{2022 -- 2024}
    \resumeSubheading
      {National University of Uzbekistan}{Tashkent, Uzbekistan}
      {B.Sc. in Information Technology}{2016 -- 2020}
  \resumeSubHeadingListEnd


%-----------RESEARCH EXPERIENCE-----------
\section{Research Experience}
  \resumeSubHeadingListStart

    \resumeSubheading
      {Student Researcher}{2022 -- Present}
      {Software Engineering Laboratory, University of Seoul}{Seoul, Korea}
      \resumeItemListStart
        \resumeItem{Built automated program repair tools using both commercial (GPT-4o) and open-source LLMs (CodeBERT, CodeLlama, Qwen 2.5 32B) with RAG and static analysis}
        \resumeItem{Designed agent-based architecture with dynamic context management, fixing 357 Java and 87 Python bugs across Defects4J and SWE-Bench Lite}
        \resumeItem{Developed classification-based fault localization achieving 74.6\% file-path accuracy on SWE-Bench Lite}
        \resumeItem{Published papers at JIPS journal, KCSE, and other conferences; co-authored paper in IEEE Access (\href{https://scholar.google.com/citations?user=W-YakaQAAAAJ&hl=en}{\underline{Google Scholar}})}
      \resumeItemListEnd

  \resumeSubHeadingListEnd


%-----------PROJECTS-----------
\section{Projects}
    \resumeSubHeadingListStart
      \resumeProjectHeading
          {\textbf{Agent-Based APR with Dynamic Context} $|$ \emph{Python, Java, GPT-4o, CodeBERT, FAISS} $|$ \href{https://github.com/soft7197/dcsa4apr}{\underline{GitHub}}}{}
          \resumeItemListStart
            \resumeItem{Multi-agent system (Context Updater, Generator, Overfitting Detector) with dynamic context pool and six static analysis tools for iterative patch refinement}
            \resumeItem{Fixed 357 bugs on Defects4J and 87 on SWE-Bench Lite, outperforming SRepair (+7.5\%), ChatRepair, and ThinkRepair}
          \resumeItemListEnd
      \resumeProjectHeading
          {\textbf{MCRepair++: Multi-Chunk Program Repair} $|$ \emph{Python, Java, PyTorch, CodeBERT} $|$ \href{https://github.com/kimjisung78/MCRepair-Plus-Plus}{\underline{GitHub}}}{}
          \resumeItemListStart
            \resumeItem{Fine-tuned CodeBERT with buggy block preprocessing and proportional patch combination for multi-chunk bugs}
            \resumeItem{Fixed 79 bugs (31 multi-chunk) on Defects4J, improving 21--342\% over TBar, CURE, and CoCoNut}
          \resumeItemListEnd
      \resumeProjectHeading
          {\textbf{Classification-Based Fault Localization} $|$ \emph{Python, GPT-4o, AST Parsing} $|$ \href{https://github.com/soft7197/cbfl}{\underline{GitHub}}}{}
          \resumeItemListStart
            \resumeItem{Classifies issue descriptions into Full/Partial/Hint categories and routes to tailored symbol-level localization strategies}
            \resumeItem{74.6\% file-path and 52.3\% symbol-level accuracy on SWE-Bench Lite, outperforming Agentless and AutoCodeRover}
          \resumeItemListEnd
    \resumeSubHeadingListEnd


%-----------TECHNICAL SKILLS-----------
\section{Technical Skills}
 \begin{itemize}[leftmargin=0.15in, label={}]
    \small{\item{
     \textbf{AI/ML}{: CodeBERT, GPT-3/4, CodeLlama, Qwen 2.5 32B, fine-tuning, prompt engineering, RAG, embedding-based retrieval} \\
     \textbf{Languages}{: Python, Java, SQL, Bash, C\#} \\
     \textbf{Frameworks \& Libraries}{: PyTorch, TensorFlow, Hugging Face Transformers, Scikit-learn, FAISS} \\
     \textbf{Tools}{: Git, Docker, Linux, OpenAI API, JavaParser, Defects4J, SWE-Bench}
    }}
 \end{itemize}


%-----------LANGUAGES-----------
\section{Languages}
 \begin{itemize}[leftmargin=0.15in, label={}]
    \small{\item{
     % TODO: Adjust CEFR levels to match your actual proficiency. Add test scores if available (IELTS, TOPIK, etc.)
     English (Advanced, B2--C1) $|$ Korean (Elementary, A2) $|$ Russian (Intermediate, B1) $|$ Uzbek (Native)
    }}
 \end{itemize}


%-------------------------------------------
\end{document}